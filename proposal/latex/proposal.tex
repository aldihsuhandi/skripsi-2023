\documentclass[a4paper]{article}


\usepackage[
    style=numeric, 
    backend=biber, 
    sorting=none
]{biblatex}
\usepackage{graphicx}
\usepackage{enumitem}
\usepackage{geometry}
\usepackage{sectsty}
\usepackage{indentfirst}
\usepackage{times}

\setlist{
  listparindent=\parindent,
  parsep=0pt,
}

\sectionfont{\centering}
\addbibresource{citation.bib}
\graphicspath{{./images/}}
\geometry{a4paper, top=4.0cm, bottom=3.0cm,
          left=4.0cm, includehead, includefoot}
\renewcommand\contentsname{Daftar Pustaka}
\emergencystretch=2em

\begin{document}
\linespread{1.5}

\title{Marketplace For Hobby}
\author{Aldih Suhandi, Chandra Wijaya, Ibrahim Seto Aditama}

\maketitle
\begin{figure}[h]
    \centering
    \includegraphics[width=10cm]{logo_binus.png}\\
    Binus University\\
    2022
\end{figure}
\begin{figure}[h]
    \centering
    Diperiksa Oleh**\\
    \vspace{15mm}
    \begin{tabular}{@{}p{2.5in}@{}}
    \centering
    Nama Dosen - Kode Dosen
    \end{tabular}
\end{figure}

\newpage
\addcontentsline{toc}{section}{\protect\numberline{}Daftar Pustaka}
\tableofcontents

\newpage
\section*{Pendahuluan}
\addcontentsline{toc}{section}{\protect\numberline{}Pendahuluan}

\newpage
\section*{Tinjauan Pustaka}
\addcontentsline{toc}{section}{\protect\numberline{}Tinjauan Pustaka}
% OOP, Functional Programming, Design Pattern, Architecural Pattern

\newpage
\section*{Metode Pelaksanaan}
\addcontentsline{toc}{section}{\protect\numberline{}Metode Pelaksanaan}

\subsection*{\textit{Tech Stack}}
\addcontentsline{toc}{subsection}{\protect\numberline{}\textit{Tech Stack}}
\begin{enumerate}
    \item \textit{Back end} 

    \textit{Back end} atau bisa disebut juga \textit{developer's end} adalah sebuah layer yang memproses semuanya dibelakang layar dan tempat terjadinya sesuatu yang tidak bisa dilihat oleh \textit{user}\autocite{letsgodojo-frontend-backend}. Analogy yang bisa digunakan adalah saat disebuah restoran ada pelanggan yang memesan sesuatu yang spesifik ke pelayan, ini adalah bagian \textit{front end} dari restoran tersebut, setelah itu pelayan memberikan pesanan tersebut ke koki, yang akan mengambil bahan masak, momotong semua bahan - bahan tersebut, dan memasaknya sesuai resep yang sudah dibuat, koki ini yang merepresentasikan bagian \textit{backend} dari restoran tersebut\autocite{codecademy-backend}. 

    \textit{Back end} terdiri dari dua hal, \textit{servers} yang akan meproses data dan \textit{request} dari sebuah user dan \textit{database} yang akan menyimpan data yang sudah atau yang akan mau diproses\autocite{codecademy-backend}. Dalam proyek ini, kami akan menggunakan \textit{springboot} sebagai \textit{framework} yang digunakan untuk membuat \textit{software back end}-nya dan \textit{\textbf{MySQL}} sebagai database yang digunakan.

    \textit{Springboot} adalah \textit{java back end framework} yang paling populer didunia, \textit{springboot} membuat menulis \textit{software back end} menjadi lebih mudah dan lebih cepat\autocite{spring-framework}, alasan kenapa kami menggunakan \textit{springboot} adalah sebagai berikut:
    \begin{itemize}
        \item \textit{Springboot} mempunyai banyak \textit{plugin} yang dapat di\textit{install};
        \item komunitas \textit{springboot} itu sangat besar;
        \item dan yang terakhir \textit{springboot} memiliki fitur \textit{Inversion of Control} dan \textit{Dependency Injection}.
    \end{itemize}

    \item \textit{Front end}
\end{enumerate}
% mysql
% springboot
% reactjs base framework
% payment API TODO: aldih

\subsection*{System Architecture}
\addcontentsline{toc}{subsection}{\protect\numberline{}System Architecture}
% sequence diagram secara general
% mock up aplikasi

\subsection*{Aplikasi Serupa}
\addcontentsline{toc}{subsection}{\protect\numberline{}Aplikasi Serupa}
% bikin table
% buat +- dari aplikasi serupa ini

\subsection*{Pengumpulan Data Pengguna}
\addcontentsline{toc}{subsection}{\protect\numberline{}Pengumpulan Data Pengguna}

\newpage
\addcontentsline{toc}{section}{\protect\numberline{}Referensi}
\printbibliography[title=Referensi]

\end{document}
