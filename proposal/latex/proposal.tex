\documentclass[a4paper]{article}

\usepackage{array}
\usepackage[
    style=numeric, 
    backend=biber, 
    sorting=none
]{biblatex}
\usepackage{graphicx}
\usepackage{enumitem}
\usepackage{geometry}
\usepackage{sectsty}
\usepackage{indentfirst}
\usepackage{times}
\usepackage{listings}
\usepackage{longtable}

\setlist{
  listparindent=\parindent,
  parsep=0pt,
}

% code block style
% -- Defining colors:
\usepackage[dvipsnames]{xcolor}
\definecolor{codegreen}{rgb}{0,0.6,0}
\definecolor{codegray}{rgb}{0.5,0.5,0.5}
\definecolor{codepurple}{rgb}{0.58,0,0.82}
\definecolor{backcolour}{rgb}{0.95,0.95,0.92}% Definig a custom style:
\lstdefinestyle{mystyle}{
    backgroundcolor=\color{backcolour},   
    commentstyle=\color{codepurple},
    keywordstyle=\color{NavyBlue},
    numberstyle=\tiny\color{codegray},
    stringstyle=\color{codepurple},
    basicstyle=\ttfamily\footnotesize\bfseries,
    breakatwhitespace=false,         
    breaklines=true,                 
    captionpos=t,                    
    keepspaces=true,                 
    numbers=left,                    
    numbersep=5pt,                  
    showspaces=false,                
    showstringspaces=false,
    showtabs=false,                  
    tabsize=2
}% -- Setting up the custom style:
\lstset{style=mystyle}

\sectionfont{\centering}
\addbibresource{citation.bib}
\graphicspath{{./images/}}
\geometry{a4paper, top=4.0cm, bottom=3.0cm,
          left=4.0cm, includehead, includefoot}
\renewcommand\contentsname{Daftar Pustaka}
\emergencystretch=2em

% bab command
\newcommand{\bab}[2]{\setcounter{section}{#1}\addtocounter{section}{-1}\section{Bab #1. #2}}

\begin{document}
\linespread{1.5}

\title{Marketplace For Hobby}
\author{Aldih Suhandi, Chandra Wijaya, Ibrahim Seto Aditama}

\maketitle
\begin{figure}[h]
    \centering
    \includegraphics[width=10cm]{logo_binus.png}\\
    Binus University\\
    2022
\end{figure}
\begin{figure}[h]
    \centering
    Diperiksa Oleh**\\
    \vspace{15mm}
    \begin{tabular}{@{}p{2.5in}@{}}
    \centering
    Nama Dosen - Kode Dosen
    \end{tabular}
\end{figure}

\newpage
\addcontentsline{toc}{section}{\protect\numberline{}Daftar Pustaka}
\tableofcontents

\newpage
\section*{Bab 1. Pendahuluan}
\addcontentsline{toc}{section}{\protect\numberline{}Bab 1. Pendahuluan}

\subsection*{1.1 Latar Belakang}
\addcontentsline{toc}{subsection}{\protect\numberline{}1.1 Latar Belakang}

\subsection*{1.2 Rumusan Masalah}
\addcontentsline{toc}{subsection}{\protect\numberline{}1.2 Rumusan Masalah}

\subsection*{1.3 Ruang Lingkup}
\addcontentsline{toc}{subsection}{\protect\numberline{}1.3 Ruang Lingkup}

\subsection*{1.4 Tujuan dan Manfaat}
\addcontentsline{toc}{subsection}{\protect\numberline{}1.4 Tujuan dan Manfaat}

\subsection*{1.5 Metode Penelitian}
\addcontentsline{toc}{subsection}{\protect\numberline{}1.5 Metode Penelitian}

\newpage
\section*{Bab 2. Tinjauan Pustaka}
\addcontentsline{toc}{section}{\protect\numberline{}Bab 2. Tinjauan Pustaka}
% OOP, Functional Programming, Design Pattern, Architecural Pattern


\subsection*{2.1 \textit{Object Oriented Programming}}
\addcontentsline{toc}{subsection}{\protect\numberline{}2.1 \textit{Object Oriented Programming}}
\textit{Object Oriented Programming} adalah konsep programming yang berdasarkan \textit{objects}, sebuah \textit{object} adalah sesuatu entitas yang ada didunia nyata yang bisa diindetifikasi secara unik\autocite{liang_liang_2021}. Sebagai contoh, sebuah meja, seorang guru, dan bahkan hutang bisa dijadikan sebagai \textit{object}. 

Untuk membuat sebuah \textit{object} diperlukan sebuah \textit{template} atau \textit{blueprint}, \textit{template} atau \textit{blueprint} ini dalam \textbf{OOP} disebut sebagai \textit{class}, secara definisi \textit{class} adalah sebuah \textit{blueprint} yang dipakai untuk membuat sesuatu atau \textit{object} yang lebih spesifik atau konkrit\autocite{education-erin-oop-2020}, \textit{class} biasanya hanya menampung atribut - atribut yang secara general, seperti tinggi, berat badan, umur, dan sebagainya. Sedangkan sebuah \textit{object} akan menampung \textit{value} yang lebih spesifik, seperti \textit{object} tersebut bertinggi 2 meter dan \textit{object} sudah berumur 3 tahun.

\textbf{OOP} memiliki 4 pilar pendukung, yaitu:
\begin{enumerate}

    \item \textit{Encapsulation}

    \textit{Encapsulation} adalah konsep dimana semua atribut atau fitur yang ada didalam \textit{object} itu tidak bisa dilihat dan diubah oleh \textit{object} lain, kecuali akses tersebut didefinisikan secara explicit\autocite{education-erin-oop-2020}. Sebagai contoh, \textit{object} manusia memiliki tanggal lahir, karena tanggal lahir itu pasti tidak bisa diubah, atribut tanggal lahir dari \textit{object} tersebut hanya bisa dibaca oleh \textit{object} lain tapi tidak bisa diubah.
        
    Kohesi, konsistensi, dan enkapsulasi adalah pedoman yang baik untuk mencapai sebuah design yang jelas. Sebuah \textit{class} harus memiliki kontrak yang jelas yang mudah dijelaskan dan dipahami, seperti atribut apa saja yang bisa diakses dan diubah oleh \textit{object} atau \textit{class} lain\autocite{liang_liang_2021}.

    \begin{figure}[h]
        \centering
        \begin{lstlisting}[language=Java]
public class Human {
    private Date dateOfBirth;

    public Date getDateOfBirth() {
        return this.dateOfBirth;
    }
}\end{lstlisting}
        \caption{Contoh dari \textit{encapsulation}}
    \end{figure}

    \item \textit{Abstraction}

    \textit{Abstraction} dan \textit{encapsulation} adalah dua sisi dari koin yang sama, \textit{encapsulation} adalah konsep untuk mengatur akses suatu atribut, kalau \textit{abstraction} adalah \textit{class} lain tidak perlu tahu bagaimana \textit{class} ini melakukan suatu tugas\autocite{liang_liang_2021}. Sebagai contoh, saat merakit komputer, kalian hanya tau komponen - komponen didalam komputer itu apa saja, untuk merakit komputer tersebut kalian hanya perlu tau apa saja peran dari tiap komponen yang ada, tidak perlu mengetahui untuk melaksanakan peran mereka, apa yang mereka harus lakukan.
    \newpage
        
    \item \textit{Inheritance}

    \textit{Inheritance} adalah sebuah konsep dimana \textit{class} dapat mempunyai atribut dan fitur turunan dari \textit{class} lain. \textit{Class} yang mendapat fitur turunan ini disebut \textit{child class}, sedangkan \textit{class} yang diturunkan disebut \textit{parent class}\autocite{education-erin-oop-2020}. 

    Tujuan dari konsep ini adalah untuk mengurangi redudansi sebanyak mungkin, dengan cara mengeneralisasikan beberapa \textit{class}, karena beberapa \textit{class} yang berbeda bisa saya memiliki fitur atau atribute yang sama. Sebagai contoh \textit{class} guru, murid, dan kepala sekolah memiliki beberapa atribut yang sama, seperti tinggi badan, berat badan, dan umur. Semua atribut yang sama tersebut bisa dijadikan sebuah \textit{parent class} yang bernama \textit{human} dan \textit{class} guru, murid, dan kepala sekolah akan menjadi \textit{child class} dari \textit{class} tersebut\autocite{liang_liang_2021}.

    \begin{figure}[h]
        \centering
        \includegraphics[width=8cm]{inheritance example.png}
        \caption{Contoh dari \textit{inheritance}}
    \end{figure}

    \item \textit{Polymorphism}

    \textit{Child class} adalah sebuah \textit{class} yang akan menuruni semua atribut dan fitur dari \textit{parent class}, tapi apabila dari satu \textit{parent class} memiliki 2 \textit{child class} yang memiliki fitur yang sama tapi cara melakukannya yang berbeda? contoh pisang goreng adalah sebuah makanan, tapi tidak semua makanan adalah pisang goreng, ada perubahan dari cara memasak dan cara penyajian ditiap - tiap makanan, disini konsep \textit{polymorphism} dapat dipakai. Secara definisi \textit{polymorphism} adalah kelakuan dimana \textit{child class} bisa melakukan suatu \textit{task} atau fitur yang sama seperti \textit{parent class} dengan cara yang berbeda\autocite{education-erin-oop-2020}.
\end{enumerate} 

    
\newpage
\subsection*{2.2 \textit{Funtional Programming}}
\addcontentsline{toc}{subsection}{\protect\numberline{}2.2 \textit{Functional Programming}}
\textit{Functional Programming} adalah sebuah konsep, paradigma atau sebuah macam \textit{software development} yang menekankan titik beratnya pada penggunaan \textit{functions}, \textit{Functional Programming} sendiri bukan merupakan sebuah alat yang dapat digunakan tetapi merupakan sesuatu pegangan untuk \textit{developers} sebagai sebuah cara menulis kode\autocite{atencio2016functional}.


Tujuan dari \textit{Functional Programming} adalah membuat sebuah \textit{function} yang lebih kecil yang memiliki sifat dapat digunakkan kembali, lebih dapat diandalkan dan mudah dimengerti, lalu \textit{functions} tersebut maka akan dapat membuat sebuah program yang lebih dapat dimengerti \autocite{atencio2016functional}. \textit{Function} yang dibuat mengikuti \textit{Functional Programming} biasa dibuat dengan melakukan parameterisasi pada \textit{function} supaya dalam menggunakannya dengan menggunakkan parameter kode tersebut dapat digunakkan kembali untuk melakukan hal yang lain. \textit{Functional Programming} memiliki empat konsep dasar, yaitu \textit{Declarative programming}, \textit{Pure functions}, \textit{Referential transparency}, dan \textit{Immutability}\autocite{atencio2016functional}.


\begin{itemize}
    \item \textit{Declarative programming} adalah sebuah pendekatan \textit{programming} dimana sebuah kode untuk melakukan sesuatu akan ditulis menggunakan \textit{expressions} yang mendeskripsikan logika dari suatu program. Berbeda dengan Imperatif atau \textit{Procedural Programming} dimana akan ditulis secara detail bagaimana untuk melakukan sesuatu untuk mencapai hasil yang diinginkan\autocite{atencio2016functional}. 
    \item \textit{Pure functions} adalah \textit{function} yang memiliki dua sifat, yaitu pertama, sebuah pure function hanya dan hanya bergantung pada input yang diberikan dan tidak dengan \textit{state} yang tersembunyi dan/atau external (diluar function tersebut) dan kedua, tidak merubah apapun yang diluar dari \textit{function} tersebut seperti obyek global atau parameter yang di \textit{pass} ke \textit{function} tersebut \autocite{atencio2016functional}. 
    \item \textit{Referential transparency} adalah ketika sebuah \textit{function} menghasilkan hasil yang sama juga diberikan input yang sama \autocite{atencio2016functional}.
    \item \textit{Immutability} adalah dimana dalam \textit{Functional Programming} untuk melestarikan sebuah data supaya \textit{Immutable} tidak bisa diganti setelah di deklarasi. Dalam konsep ini yang perlu diperhatikan adalah object seperti \textit{array} yang dapat berubah konten atau valuenya dalam sebuah function \autocite{atencio2016functional}.  
\end{itemize}

\subsection*{2.3 \textit{Architectural Design}}
\addcontentsline{toc}{subsection}{\protect\numberline{}2.3 \textit{Architectural Design}}

\subsection*{2.4 \textit{Design Pattern}}
\addcontentsline{toc}{subsection}{\protect\numberline{}2.4 \textit{Design Pattern}}

\textit{Design Pattern} secara definisi adalah sesuatu cara komunikasi tiap obyek dan \textit{class} yang dikostumisasi untuk memecahkan masalah desain general didalam konteks tertentu\autocite{design-pattern-2588942}.

\newpage
\section*{Bab 3. Metode Pelaksanaan}
\addcontentsline{toc}{section}{\protect\numberline{}Bab 3. Metode Pelaksanaan}
\subsection*{3.1 Metode Pelaksanaan}
\addcontentsline{toc}{subsection}{\protect\numberline{}3.1 Metode Pelaksanaan}


Pengumpulan data yang kami lakukan ada lah dengan menggunakkan \textit{Google Form} membuat form survei untuk mendapatkan pendapat atas pertanyaan-pertanyaan kami terkait dengan alasan dari pembuatan aplikasi pada proyek ini, yaitu seputar pembelian barang hobi dan tingkat kesulitan dalam menemukan barang yang cocok. Dari jawaban dalam survei tersebut maka kita dapat memperoleh gambaran dan konfirmasi atas fitur-fitur yang kami butuhkan.


Pertanyaan “Apakah anda takut untuk mengambil hobi baru karena tidak tau mulai dari mana?”, kami ajukan kepada pengisi survei untuk mendapatkan data dan konfirmasi apakah aplikasi dalam proyek kami dibutuhkan, dan dengan hasil jawaban mayoritas diatas 60\% menjawab iya, kami mendapatkan data bahwa terdapat kesulitan memulai hobi karena keraguan titik mulai hobi tersebut.


Selanjutnya dengan “Apakah anda sering menggunakan \textit{e-commerce} untuk browsing dan mencari barang berhubungan dengan hobi anda?” untuk validasi apakah menggunakkan fasilitas online untuk membeli barang hobi itu merupakan metode yang populer. Dengan masukan yang kami dapat diatas 70\% menjawab iya maka kami mendapatkan bahwa fasilitas jual/beli barang hobi di internet itu layak dilakukan.


Untuk pertanyaan “Dalam mencari dan menemukan barang apakah anda terkadang merasa ragu dan harus mencari tahu terlebih dahulu atas barang yang anda lihat apakah untuk pemula/menengah/ahli?” ini adalah untuk mengumpulkan data atas fitur utama yang kami rancang untuk aplikasi ini yaitu fitur filtrasi level kemahiran (\textit{interest level}) dari barang hobi. Dengan respons yang berjawab iya diatas 70\% maka fitur utama kami untuk aplikasi itu layak untuk diimplementasikan.

\subsection*{3.2 Analisis}
\addcontentsline{toc}{subsection}{\protect\numberline{}3.2 Analisis}

\subsubsection*{3.2.1 Analisis perbandingan aplikasi sejenis}
\addcontentsline{toc}{subsubsection}{\protect\numberline{}3.2.1 Analisis perbandingan aplikasi sejenis}
Terdapat dua aplikasi serupa dengan aplikasi didalam proposal proyek ini yang kami jadi obyek oberservasi, yaitu \textit{Reddit} dan \textit{Tokopedia}.

\begin{longtable}{|m{2cm}|p{5cm}|p{5cm}|}
    % \begin{tabular}{| c | c | c |} 
    \hline
    Aplikasi & Positif & Negatif \\ 
    \hline
    Reddit 
    &   + \textit{community} yang sudah besar \newline 
        + System handling \textit{post} dan komen sudah \textit{robust} \newline 
        + Interaksi antara user sudah bagus 
    &   - Bukan marketplace \newline 
        - Tidak cocok untuk memberi rating konkrit kepada barang \newline 
        - Tidak ada gambar atau suatu panel untuk menampilkan informasi item tersebut secara singkat \\ 
    \hline
    Tokopedia
    &   + Proses shipping dan payment sudah berbagai macam \newline  
        + Banyak promo yang bisa didapatkan user  
    &   - Tidak bisa mengelompokkan item berdasarkan hobi \newline 
        - User akan sulit mencari item yang cocok bagi mereka karena ketidakadaannya interest level untuk item tersebut \newline 
        - Tidak ada forum dedikasi diskusi, hanya untuk diskusi barang \\ 
    \hline
\end{longtable}


\subsubsection*{3.2.2 Analisis permasalah}
\addcontentsline{toc}{subsubsection}{\protect\numberline{}3.2.2 Analisis permasalah}

\subsubsection*{3.2.3 Usulan pemecahan masalah}
\addcontentsline{toc}{subsubsection}{\protect\numberline{}3.2.3 Usulan pemecahan masalah}


\subsection*{3.3 Perancangan}
\addcontentsline{toc}{subsection}{\protect\numberline{}3.3 Perancangan}

\subsubsection*{3.3.1 \textit{Software Design Document}}
\addcontentsline{toc}{subsubsection}{\protect\numberline{}3.3.1 \textit{Software Design Document}}
\begin{enumerate}[label=\alph*. ]
    \item Deskripsi \textit{software}
    
    \textit{Shumishumi} adalah sebuah marketplace dimana user bisa membeli suatu barang sesuai dengan hobi dan setinggi apa \textit{interest level} mereka. Tujuan dari aplikasi ini adalah mempermudah pengguna pembeli untuk mencari barang hobi yang cocok, mempermudah penjual dalam mempromosikan barang-nya kepada pelanggan dengan fitur filter \textit{Interest Level}. 


    \textit{Shumishumi} juga bisa menjadi sarana untuk berdiskusi mengenai topik dan hobi yang diminati, dengan fitur ini kami berharap dapa mempermudah \textit{user} mencari dan memasuki hobi baru.

    \item Fungsi-fungsi \textit{software}

    % sequence diagram

    \item Kebutuhan teknologi

    \begin{enumerate}
        \item \textit{Back end} 

        \textit{Back end} atau bisa disebut juga \textit{developer's end} adalah sebuah layer yang memproses semuanya dibelakang layar dan tempat terjadinya sesuatu yang tidak bisa dilihat oleh \textit{user}\autocite{letsgodojo-frontend-backend}. Analogy yang bisa digunakan adalah saat disebuah restoran ada pelanggan yang memesan sesuatu yang spesifik ke pelayan, ini adalah bagian \textit{front end} dari restoran tersebut, setelah itu pelayan memberikan pesanan tersebut ke koki, yang akan mengambil bahan masak, momotong semua bahan - bahan tersebut, dan memasaknya sesuai resep yang sudah dibuat, koki ini yang merepresentasikan bagian \textit{backend} dari restoran tersebut\autocite{codecademy-backend}. 

        \textit{Back end} terdiri dari dua hal, \textit{servers} yang akan meproses data dan \textit{request} dari sebuah user dan \textit{database} yang akan menyimpan data yang sudah atau yang akan mau diproses\autocite{codecademy-backend}. Dalam proyek ini, kami akan menggunakan \textit{springboot} sebagai \textit{framework} yang digunakan untuk membuat \textit{software back end}-nya dan \textit{\textbf{MySQL}} sebagai database yang digunakan.

        \textit{Springboot} adalah \textit{java back end framework} yang paling populer didunia, \textit{springboot} membuat menulis \textit{software back end} menjadi lebih mudah dan lebih cepat\autocite{spring-framework}, alasan kenapa kami menggunakan \textit{springboot} adalah sebagai berikut:
        \begin{itemize}
            \item \textit{Springboot} mempunyai banyak \textit{plugin} yang dapat di\textit{install};
            \item komunitas \textit{springboot} itu sangat besar;
            \item dan yang terakhir \textit{springboot} memiliki fitur \textit{Inversion of Control} dan \textit{Dependency Injection}.
        \end{itemize}

        \item \textit{Front end}
    \end{enumerate}

\end{enumerate}


\newpage
\subsubsection*{3.3.2 Perancangan Sistem}
\addcontentsline{toc}{subsubsection}{\protect\numberline{}3.3.2 Perancangan Sistem}

\newpage
\addcontentsline{toc}{section}{\protect\numberline{}Referensi}
\printbibliography[title=Referensi]

\end{document}
