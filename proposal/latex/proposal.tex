\documentclass[a4paper]{article}


\usepackage[
    style=numeric, 
    backend=biber, 
    sorting=none
]{biblatex}
\usepackage{graphicx}
\usepackage{enumitem}
\usepackage{geometry}
\usepackage{sectsty}
\usepackage{indentfirst}
\usepackage{times}

\setlist{
  listparindent=\parindent,
  parsep=0pt,
}

\sectionfont{\centering}
\addbibresource{citation.bib}
\graphicspath{{./images/}}
\geometry{a4paper, top=4.0cm, bottom=3.0cm,
          left=4.0cm, includehead, includefoot}
\renewcommand\contentsname{Daftar Pustaka}
\emergencystretch=2em

\begin{document}
\linespread{1.5}

\title{Marketplace For Hobby}
\author{Aldih Suhandi, Chandra Wijaya, Ibrahim Seto Aditama}

\maketitle
\begin{figure}[h]
    \centering
    \includegraphics[width=10cm]{logo_binus.png}\\
    Binus University\\
    2022
\end{figure}
\begin{figure}[h]
    \centering
    Diperiksa Oleh**\\
    \vspace{15mm}
    \begin{tabular}{@{}p{2.5in}@{}}
    \centering
    Nama Dosen - Kode Dosen
    \end{tabular}
\end{figure}

\newpage
\addcontentsline{toc}{section}{\protect\numberline{}Daftar Pustaka}
\tableofcontents

\newpage
\section*{Pendahuluan}
\addcontentsline{toc}{section}{\protect\numberline{}Pendahuluan}

Perkembangan teknologi yang pesat telah melahirkan banyak teknologi baru yang dapat membantu dan mempermudah kita manusia dalam menjalankan kehidupan dan melaksanakan tugas dan keinginan kita. Dari perkembangan teknologi tersebut, teknologi yang memiliki dampak yang cukup tinggi adalah internet. Yang dimana di negara Indonesia ini internet mendapatkan peningkatan popularitas yang dapat diukur dengan peningkatan pengguna internet tersebut dalam tiap tahun, Menurut survey yang diadakan oleh Asosiasi Penyelenggara Jasa Internet Indonesia (APJII)  di tahun 2017 perkembangan pengguna internet dari tahun 2016 ke tahun 2017 memiliki pertambahan sebesar 10.56 juta (disurvey pada 2016 ada 132.7 juta dan di 2017 ada 143,26 juta)\autocite{indonesia2017infografis}.


Dari penggunaan internet, kegiatan usaha jual/beli barang yang dilakukan secara text/chat ataupun melalui \textit{e-commerce} merupakan kegiatan yang cukup populer saat memanfaatkan internet dimana di Indonesia di survey APJII pembelian barang memiliki 32.19\% dan jual barang 8.12\% pada layanan yang diakses ketika mengakses internet \autocite{indonesia2017infografis}. \textit{Platform} untuk mempermudah usaha kegiatan jual/beli barang dilengkapi fitur \textit{search} yang cukup, dimana barang dapat dicari dan ditemukan melalui mencari namanya atau menelusuri kategori dari barang tersebut. Kegiatan beli barang dapat untuk mendapatkan barang baku, kebutuhan sehari-hari dan lain-lain. Kategori barang yang menjadi perhatian dan sorotan kami adalah barang berhubungan dengan hobi dimana yang mengalami peningkatan popularitas belakangan ini oleh karena pandemi\autocite{langstedt2022loneliness}.


Oleh karena itu kami melakukan survei untuk mengetahui tentang apa yang mungkin menjadi kesulitan dalam membeli barang terkait hobi di aplikasi \textit{e-commerce}, yang dimana kami mendapatkan bahwa untuk pertanyaan kami “Apakah anda takut untuk mengambil hobi baru karena tidak tau mulai dari mana?” memiliki respon 66.7\% menjawab iya, dan juga pda pertanyaan berhubungan mengenai kesulitan untuk memilih barang yang cocok memiliki respons ya pada persentase yang diatas 50%.


Dari latar belakang tersebut, dimana dalam membeli sebuah barang hobi terjadi keraguan/kesulitan dalam memilih barang yang cocok untuk memulai atau membeli yang cocok, tujuan kami adalah untuk mengajukan sebuah sistem aplikasi jual/beli barang khusus dan terfokus mengenai Hobi-hobi dimana barang yang dijual selain dapat diberi kategori, barang juga dapat diberikan dua penilaian tingkat peminatan dalam hobi tersebut sebagai filter tambahan, kita beri nama \textit{Interest Level}, penilaian tersebut diberikan oleh penjual dan juga nanti diberi oleh pembeli sebagai validasi apakah pemberian penilaian tersebut sesuai atau tidak. Dan juga karena ini sebuah aplikasi untuk hobby, ada sebuah forum untuk melakukan diskusi dengan cara membuat post mengenai kategori hobi yang diinginkan.

\newpage
\section*{Tinjauan Pustaka}
\addcontentsline{toc}{section}{\protect\numberline{}Tinjauan Pustaka}
% OOP, Functional Programming, Design Pattern, Architecural Pattern

\newpage
\section*{Metode Pelaksanaan}
\addcontentsline{toc}{section}{\protect\numberline{}Metode Pelaksanaan}

\subsection*{\textit{Tech Stack}}
\addcontentsline{toc}{subsection}{\protect\numberline{}\textit{Tech Stack}}
% mysql
% springboot
% reactjs base framework
% payment API TODO: aldih

\subsection*{System Architecture}
\addcontentsline{toc}{subsection}{\protect\numberline{}System Architecture}
% usecase diagram
% activity diagram
% sequence diagram secara general
% mock up aplikasi

\subsection*{Aplikasi Serupa}
\addcontentsline{toc}{subsection}{\protect\numberline{}Aplikasi Serupa}
% bikin table
% buat +- dari aplikasi serupa ini

\subsection*{Pengumpulan Data Pengguna}
\addcontentsline{toc}{subsection}{\protect\numberline{}Pengumpulan Data Pengguna}

\newpage
\addcontentsline{toc}{section}{\protect\numberline{}Referensi}
\printbibliography[title=Referensi]

\end{document}
