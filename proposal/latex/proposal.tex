\documentclass[a4paper]{article}

\usepackage{array}
\usepackage[
    style=apa, 
    backend=biber, 
    sorting=none
]{biblatex}
\usepackage{graphicx}
\usepackage{enumitem}
\usepackage{geometry}
\usepackage{sectsty}
\usepackage{indentfirst}
\usepackage{times}
\usepackage{listings}
\usepackage{longtable}
\usepackage{setspace}
\usepackage[colorlinks=true,linkcolor=black,anchorcolor=black,citecolor=black,filecolor=black,menucolor=black,runcolor=black,urlcolor=black]{hyperref}

\setlist{
  listparindent=\parindent,
  parsep=0pt,
}

% code block style
% -- Defining colors:
\usepackage[dvipsnames]{xcolor}
\definecolor{codegreen}{rgb}{0,0.6,0}
\definecolor{codegray}{rgb}{0.5,0.5,0.5}
\definecolor{codepurple}{rgb}{0.58,0,0.82}
\definecolor{backcolour}{rgb}{0.95,0.95,0.92}% Definig a custom style:
\lstdefinestyle{mystyle}{
    backgroundcolor=\color{backcolour},   
    commentstyle=\color{codepurple},
    keywordstyle=\color{NavyBlue},
    numberstyle=\tiny\color{codegray},
    stringstyle=\color{codepurple},
    basicstyle=\ttfamily\footnotesize\bfseries,
    breakatwhitespace=false,         
    breaklines=true,                 
    captionpos=t,                    
    keepspaces=true,                 
    numbers=left,                    
    numbersep=5pt,                  
    showspaces=false,                
    showstringspaces=false,
    showtabs=false,                  
    tabsize=2
}% -- Setting up the custom style:
\lstset{style=mystyle}

\sectionfont{\centering}
\addbibresource{citation.bib}
\graphicspath{{./images/}}
\geometry{a4paper, top=4.0cm, bottom=3.0cm,
          left=4.0cm, includehead, includefoot}
\renewcommand\contentsname{Daftar Pustaka}
\emergencystretch=2em

% bab command
\newcommand{\bab}[2]{\setcounter{section}{#1}\addtocounter{section}{-1}\section{Bab #1. #2}}

\begin{document}
\setstretch{1.5}

\title{PROPOSAL SKRIPSI NON KELAS\\Shumishumi - Marketplace For Hobby With Interest Level Filtering\large\\\textbf{Topic: }\textit{E-Application}}
\author{2301862632 - Aldih Suhandi - \textit{Computer Science} / 082122704561\\2301872596 - Chandra Wijaya - \textit{Computer Science} / 081287588816\\2301867324 - Ibrahim Seto Aditama - \textit{Computer Science} / 081213423549}


\maketitle
\begin{figure}[h]
    \centering
    \includegraphics[width=10cm]{logo_binus.png}\\
    Binus University Alam Sutera\\
    2022
\end{figure}
\begin{figure}[h]
    \centering
    Diperiksa Oleh**\\
    \vspace{15mm}
    \begin{tabular}{@{}p{2.5in}@{}}
        \centering
        Nama Dosen - Kode Dosen
    \end{tabular}
\end{figure}

\newpage
\addcontentsline{toc}{section}{\protect\numberline{}Daftar Pustaka}
\tableofcontents

\newpage
\section*{Bab 1. Pendahuluan}
\addcontentsline{toc}{section}{\protect\numberline{}Bab 1. Pendahuluan}

\subsection*{1.1 Latar Belakang}
\addcontentsline{toc}{subsection}{\protect\numberline{}1.1 Latar Belakang}

Hobi memiliki pengertian sebagai suatu aktivitas yang dilakukan untuk mendapatkan rasa bahagia, yang dilakukan dalam waktu luang. Hobi juga dapat menjadi distraksi dari rasa gelisah, dan juga ketika sebuah hobi dilakukan dengan kelompok yang memiliki hobi yang sama, hobi dapat memberi rasa tujuan hidup\autocite{zaidi2022passion}.


Untuk beberapa aktivitas hobi, ada barang yang diperlukan dalam menjalani aktivitas tersebut, sebagai contoh, untuk hobi memancing membutuhkan alat pancing. Untuk mencari dan mendapatkan barang tersebut ada dengan cara mendatangi toko khusus untuk hobi tersebut, atau pun mencarinya lewat internet.


Internet merupakan hasil dari perkembangan teknologi yang memiliki dampak cukup tinggi, dimana di Indonesia sendiri, internet mendapatkan peningkatan popularitas yang cukup drastis. Hal ini dapat diukur dengan melihat peningkatan pengguna internet tersebut di setiap tahunnya. Menurut survey yang diadakan oleh Asosiasi Penyelenggara Jasa Internet Indonesia (APJII) di tahun 2017, perkembangan pengguna internet dari tahun 2016 ke tahun 2017 memiliki pertambahan sebesar 10.56 juta (disurvey pada 2016 ada 132.7 juta dan di 2017 ada 143,26 juta)\autocite{indonesia2017infografis}.


Dari penggunaan internet, kegiatan usaha jual beli barang yang dilakukan secara \textit{text/chat} ataupun melalui \textit{e-commerce} merupakan kegiatan yang cukup populer saat memanfaatkan internet dimana di Indonesia di survey APJII pembelian barang memiliki 32.19\% dan jual barang 8.12\% pada layanan yang diakses ketika mengakses internet\autocite{indonesia2017infografis}. \textit{Platform} untuk mempermudah usaha kegiatan jual beli barang dilengkapi fitur \textit{search} yang cukup, dimana barang dapat dicari dan ditemukan melalui mencari namanya atau menelusuri kategori dari barang tersebut. Kegiatan beli barang dapat untuk mendapatkan barang baku, kebutuhan sehari-hari dan lain-lain.


Salah satu tempat dimana kegiatan jual beli tersebut berlangsung adalah dalam sebuah \textit{Web Application} atau aplikasi web, yang dimana aplikasi web itu sendiri adalah sebuah program aplikasi yang di simpan di sebuah \textit{remote server} dan dapat diakses dengan \textit{web browser} melalui internet\autocite{what-is-web-app}.


Kegiatan jual beli barang hobi juga tentunya termasuk dalam barang yang diperjualbelikan di tempat \textit{e-commerce} atau \textit{marketplace} di internet. Dari tempat-tempat tersebut barang-barang hobi dijual bersamaan dengan barang lainnya, ataupun terfokus kepada hobi tertentu, yang disebut \textit{Hobby shop}. Dari tempat-tempat jual beli tersebut, general maupun terfokus kepada suatu hobi, penelitian ini mengusulkan untuk membuat sebuah aplikasi web khusus untuk hobi apapun, lebih spesifik ke hobi dibanding \textit{e-commerce} atau \textit{marketplace} namun lebih luas dibanding \textit{Hobby shop} yang hanya spesifik pada satu hobi tertentu.


Penelitian ini juga bermaksud untuk mengimplementasikan fitur yang khusus dan hanya cocok di aplikasi web jual beli berbagai barang hobi, dengan melakukan survei terhadap kesulitan melakukan kegiatan jual beli barang hobi sekarang. Salah satu dari pertanyaan survei adalah apakah anda takut untuk mengambil hobi baru karena tidak tau mulai dari mana?” memiliki respon 66.7\% menjawab iya, dan juga pda pertanyaan berhubungan mengenai kesulitan untuk memilih barang yang cocok memiliki respons ya pada persentase yang diatas 50\%. Jawaban ya yang tinggi pada pertanyaan tersebut menjadi sebuah latar belakang dari fitur utama aplikasi ini yaitu, dua penilaian tingkat peminatan dalam hobi tersebut sebagai filter tambahan, diberi nama \textit{Interest Level}, penilaian tersebut diberikan oleh penjual dan juga nanti diberi oleh pembeli sebagai validasi apakah pemberian penilaian tersebut sesuai atau tidak, untuk membantu meningkatkan keyakinan seseorang untuk memilih barang hobi yang tepat dan memulai sebuah hobi.

Selain dari data-data yang sudah dikumpulkan diatas, ada juga beberapa aplikasi \textit{marketplace} sejenis dari penelitian sebelumnya yang dapat dibandingkan dengan aplikasi yang akan dikerjakan pada proyek ini. Seperti aplikasi pada penelitian yang dilakukan oleh Heru Nugroho dengan rekan-rekannya yaitu sebuah aplikasi \textit{marketplace} berbasis \textit{mobile} yang menjual belikan produk-produk agrikultur. Aplikasi tersebut bertujuan untuk mempermudah para petani/pekebun untuk menjual belikan produk agrikultur mereka ke masyarakat umum\autocite{agriculture-marketplace}. Selain sektor agrikultur, terdapat juga penelitian yang membuat aplikasi sejenis di sektor perikanan. Yaitu, penelitian yang dilakukan oleh Mohammad Nazrul bersama dengan rekan-rekannya yang membuat aplikasi \textit{marketplace} berbasis \textit{mobile} yang menjual belikan ikan\autocite{fishes-marketplace}. Lalu, ada pada ruang lingkup yang sama, yakni penelitian yang dilakukan oleh Dwi bersama dengan rekan-rekannya yang membuat aplikasi jual beli ikan menggunakan metode \textit{SMS Gateway}\autocite{c2c-fish-marketplace}. Selanjutnya, penelitian yang dilakukan oleh N F Rozi dan rekan-rekannya dengan membuat aplikasi berbasis \textit{web} bernama \textbf{LIDI} yang ditujukan sebagai tempat rental mobil secara \textit{online}\autocite{lidi-car-rental}. Terakhir, penelitian yang dilakukan oleh Rachmadita dan rekan-rekannya yang membuat sebuah aplikasi \textit{marketplace} yang menggabungkan usaha mikro, kecil dan menengah ke dalam satu \textit{platform} yang sama secara \textit{online} menggunakan metode \textit{Rapid Application Development}. Penelitian ini ditujukan kepada Badan Usaha Milik Desa \textbf{(BUM)} di wilayah Desa Mekarsari, Bandung, Jawa Barat\autocite{bum-mekarsari}. Dan ada dua aplikasi pasaran berbasis \textit{web} yang juga dapat dibandingkan dengan aplikasi yang akan dibuat pada proyek ini, yaitu satu aplikasi \textit{marketplace hobby shop} bernama \textbf{KyouID} dan aplikasi forum diskusi bernama \textbf{Reddit}.

Dari gambaran dan data yang dikumpulkan diatas, dimana adanya aktivitas jual beli yang besar melalui internet, dan kesulitan dalam memulai hobi karena tidak yakin atas barang yang cocok sebagai titik mula, maka penelitian ini bermaksud untuk merancang dan membuat sebuah aplikasi web yang khusus dan terfokus untuk jual beli barang terkait hobi dengan fitur penilaian dari penjual dan pembeli pada suatu barang terhadap \textit{Interest Level} yang cocok dan juga sebuah forum untuk melakukan diskusi dengan cara membuat post mengenai kategori hobi yang diinginkan.


\subsection*{1.2 Rumusan Masalah}
\addcontentsline{toc}{subsection}{\protect\numberline{}1.2 Rumusan Masalah}

Sesuai dengan latar belakang, rumusan masalah yang akan dikaji dalam proyek ini adalah bagaimana merancang dan membuat sebuah aplikasi \textit{marketplace} untuk jual beli barang hobi dengan fitur filter \textit{Interest Level} atau tingkat peminatan dari tiap barang tersebut dan juga sebuah forum diskusi mengenai hobi yang berada di web dan berbentuk aplikasi web.

\subsection*{1.3 Ruang Lingkup}
\addcontentsline{toc}{subsection}{\protect\numberline{}1.3 Ruang Lingkup}
Ruang lingkup dari proyek ini adalah membuat suatu \textit{marketplace} dimana seseorang bisa membeli sebuah barang yang berkaitan dengan hobi tertentu, fitur - fitur dari \textit{marketplace} ini dirancang untuk membuat memilih barang saat ingin memasuki hobi atau saat ingin lebih mendalami suatu hobi, selain menjadi \textit{marketplace} proyek juga ingin menyediakan forum dimana semua pengguna bisa bertanya, berdiskusi, ataupun menyampaikan keluh kesah dari hobi mereka.

\subsection*{1.4 Tujuan dan Manfaat}
\addcontentsline{toc}{subsection}{\protect\numberline{}1.4 Tujuan dan Manfaat}

Tujuan dari proyek ini adalah untuk membuat sebuah aplikasi \textit{marketplace} untuk jual beli barang hobi. Yang memiliki manfaat, mempermudah pengguna untuk mencari barang hobi yang cocok. Mempermudah penjual dalam mempromosikan barang-nya kepada pelanggan melalui fitur filter \textit{Interest Level}. Menyediakan sarana diskusi mengenai hobi melewati forum.

\subsection*{1.5 Metode Penelitian}
\addcontentsline{toc}{subsection}{\protect\numberline{}1.5 Metode Penelitian}

Metode Penelitian yang digunakkan adalah metode penelitian kualitatif, yang bersifat deskriptif dengan melakukan analisis\autocite{pengajar-kualitatif}, dan akan melakukan analisis dari hasil pengumpulan data yang dilakukan.


Pengumpulan data yang dilakukan adalah dengan membuat form survei menggunakan \textit{Google Form} untuk mendapatkan pendapat atas pertanyaan-pertanyaan yang terkait dengan alasan dari pembuatan aplikasi pada proyek ini. Dari hal mengenai pembelian barang hobi hingga tingkat kesulitan dalam menemukan barang yang cocok. Dari semua jawaban survei yang dikumpulkan tersebutlah, dapat memperoleh gambaran dan konfirmasi atas fitur-fitur yang proyek ini butuhkan.

\newpage
\section*{Bab 2. Tinjauan Pustaka}
\addcontentsline{toc}{section}{\protect\numberline{}Bab 2. Tinjauan Pustaka}
% OOP, Functional Programming, Design Pattern, Architecural Pattern


\subsection*{2.1 \textit{Object Oriented Programming}}
\addcontentsline{toc}{subsection}{\protect\numberline{}2.1 \textit{Object Oriented Programming}}
\textit{Object Oriented Programming} adalah konsep programming yang berdasarkan \textit{objects}, sebuah \textit{object} adalah sesuatu entitas yang ada didunia nyata yang bisa diindetifikasi secara unik\autocite{liang_liang_2021}. Sebagai contoh, sebuah meja, seorang guru, dan bahkan hutang bisa dijadikan sebagai \textit{object}.

Untuk membuat sebuah \textit{object} diperlukan sebuah \textit{template} atau \textit{blueprint}, \textit{template} atau \textit{blueprint} ini dalam \textbf{OOP} disebut sebagai \textit{class}, secara definisi \textit{class} adalah sebuah \textit{blueprint} yang dipakai untuk membuat sesuatu atau \textit{object} yang lebih spesifik atau konkrit\autocite{education-erin-oop-2020}, \textit{class} biasanya hanya menampung atribut - atribut yang secara general, seperti tinggi, berat badan, umur, dan sebagainya. Sedangkan sebuah \textit{object} akan menampung \textit{value} yang lebih spesifik, seperti \textit{object} tersebut bertinggi 2 meter dan \textit{object} sudah berumur 3 tahun.

\textbf{OOP} memiliki 4 pilar pendukung, yaitu:
\begin{enumerate}

    \item \textit{Encapsulation}

          \textit{Encapsulation} adalah konsep dimana semua atribut atau fitur yang ada didalam \textit{object} itu tidak bisa dilihat dan diubah oleh \textit{object} lain, kecuali akses tersebut didefinisikan secara explicit\autocite{education-erin-oop-2020}. Sebagai contoh, \textit{object} manusia memiliki tanggal lahir, karena tanggal lahir itu pasti tidak bisa diubah, atribut tanggal lahir dari \textit{object} tersebut hanya bisa dibaca oleh \textit{object} lain tapi tidak bisa diubah.

          Kohesi, konsistensi, dan enkapsulasi adalah pedoman yang baik untuk mencapai sebuah design yang jelas. Sebuah \textit{class} harus memiliki kontrak yang jelas yang mudah dijelaskan dan dipahami, seperti atribut apa saja yang bisa diakses dan diubah oleh \textit{object} atau \textit{class} lain\autocite{liang_liang_2021}.

          \begin{figure}[h]
              \centering
              \begin{lstlisting}[language=Java]
public class Human {
    private Date dateOfBirth;

    public Date getDateOfBirth() {
        return this.dateOfBirth;
    }
}\end{lstlisting}
              \caption{Contoh dari \textit{encapsulation}}
          \end{figure}
          \newpage

    \item \textit{Abstraction}

          \textit{Abstraction} dan \textit{encapsulation} adalah dua sisi dari koin yang sama, \textit{encapsulation} adalah konsep untuk mengatur akses suatu atribut, kalau \textit{abstraction} adalah \textit{class} lain tidak perlu tahu bagaimana \textit{class} ini melakukan suatu tugas\autocite{liang_liang_2021}. Sebagai contoh, saat merakit komputer, kalian hanya tau komponen - komponen didalam komputer itu apa saja, untuk merakit komputer tersebut kalian hanya perlu tau apa saja peran dari tiap komponen yang ada, tidak perlu mengetahui untuk melaksanakan peran mereka, apa yang mereka harus lakukan.

    \item \textit{Inheritance}

          \textit{Inheritance} adalah sebuah konsep dimana \textit{class} dapat mempunyai atribut dan fitur turunan dari \textit{class} lain. \textit{Class} yang mendapat fitur turunan ini disebut \textit{child class}, sedangkan \textit{class} yang diturunkan disebut \textit{parent class}\autocite{education-erin-oop-2020}.

          Tujuan dari konsep ini adalah untuk mengurangi redudansi sebanyak mungkin, dengan cara mengeneralisasikan beberapa \textit{class}, karena beberapa \textit{class} yang berbeda bisa saya memiliki fitur atau atribute yang sama. Sebagai contoh \textit{class} guru, murid, dan kepala sekolah memiliki beberapa atribut yang sama, seperti tinggi badan, berat badan, dan umur. Semua atribut yang sama tersebut bisa dijadikan sebuah \textit{parent class} yang bernama \textit{human} dan \textit{class} guru, murid, dan kepala sekolah akan menjadi \textit{child class} dari \textit{class} tersebut\autocite{liang_liang_2021}.

          \begin{figure}[h]
              \centering
              \includegraphics[width=8cm]{inheritance example.png}
              \caption{Contoh dari \textit{inheritance}}
          \end{figure}

    \item \textit{Polymorphism}

          \textit{Child class} adalah sebuah \textit{class} yang akan menuruni semua atribut dan fitur dari \textit{parent class}, tapi apabila dari satu \textit{parent class} memiliki 2 \textit{child class} yang memiliki fitur yang sama tapi cara melakukannya yang berbeda? contoh pisang goreng adalah sebuah makanan, tapi tidak semua makanan adalah pisang goreng, ada perubahan dari cara memasak dan cara penyajian ditiap - tiap makanan, disini konsep \textit{polymorphism} dapat dipakai. Secara definisi \textit{polymorphism} adalah kelakuan dimana \textit{child class} bisa melakukan suatu \textit{task} atau fitur yang sama seperti \textit{parent class} dengan cara yang berbeda\autocite{education-erin-oop-2020}.
\end{enumerate}


% \newpage
\subsection*{2.2 \textit{Functional Programming}}
\addcontentsline{toc}{subsection}{\protect\numberline{}2.2 \textit{Functional Programming}}
\textit{Functional Programming} adalah sebuah konsep, paradigma atau sebuah macam \textit{software development} yang menekankan titik beratnya pada penggunaan \textit{functions}, \textit{Functional Programming} sendiri bukan merupakan sebuah alat yang dapat digunakan tetapi merupakan sesuatu pegangan untuk \textit{developers} sebagai sebuah cara menulis kode\autocite{atencio2016functional}.


Tujuan dari \textit{Functional Programming} adalah membuat sebuah \textit{function} yang lebih kecil yang memiliki sifat dapat digunakkan kembali, lebih dapat diandalkan dan mudah dimengerti, lalu \textit{functions} tersebut maka akan dapat membuat sebuah program yang lebih dapat dimengerti \autocite{atencio2016functional}. \textit{Function} yang dibuat mengikuti \textit{Functional Programming} biasa dibuat dengan melakukan parameterisasi pada \textit{function} supaya dalam menggunakannya dengan menggunakkan parameter kode tersebut dapat digunakkan kembali untuk melakukan hal yang lain. \textit{Functional Programming} memiliki empat konsep dasar, yaitu \textit{Declarative programming}, \textit{Pure functions}, \textit{Referential transparency}, dan \textit{Immutability}\autocite{atencio2016functional}.


\begin{itemize}
    \item \textit{Declarative programming} adalah sebuah pendekatan \textit{programming} dimana sebuah kode untuk melakukan sesuatu akan ditulis menggunakan \textit{expressions} yang mendeskripsikan logika dari suatu program. Berbeda dengan Imperatif atau \textit{Procedural Programming} dimana akan ditulis secara detail bagaimana untuk melakukan sesuatu untuk mencapai hasil yang diinginkan\autocite{atencio2016functional}.
    \item \textit{Pure functions} adalah \textit{function} yang memiliki dua sifat, yaitu pertama, sebuah pure function hanya dan hanya bergantung pada input yang diberikan dan tidak dengan \textit{state} yang tersembunyi dan/atau external (diluar function tersebut) dan kedua, tidak merubah apapun yang diluar dari \textit{function} tersebut seperti obyek global atau parameter yang di \textit{pass} ke \textit{function} tersebut \autocite{atencio2016functional}.
    \item \textit{Referential transparency} adalah ketika sebuah \textit{function} menghasilkan hasil yang sama juga diberikan input yang sama \autocite{atencio2016functional}.
    \item \textit{Immutability} adalah dimana dalam \textit{Functional Programming} untuk melestarikan sebuah data supaya \textit{Immutable} tidak bisa diganti setelah di deklarasi. Dalam konsep ini yang perlu diperhatikan adalah object seperti \textit{array} yang dapat berubah konten atau valuenya dalam sebuah function \autocite{atencio2016functional}.
\end{itemize}

\subsection*{2.3 \textit{Architectural Pattern}}
\addcontentsline{toc}{subsection}{\protect\numberline{}2.3 \textit{Architectural Pattern}}
\textit{Architectural Pattern} merupakan seperangkat prinsip dan pola kasar yang menyediakan kerangka kerja abstrak dari sebuah sistem. \textit{Architectural Pattern} menggambarkan struktural fundamental dari suatu organisasi atau skema untuk sebuah sistem yang kompleks. \textit{Architectural Pattern} dapat digunakan sebagai solusi umum yang dapat digunakan kembali \textit{(reusable)} untuk memecahkan permasalahan yang biasa terjadi dalam sebuah \textit{software architecture} didalam konteks tertentu\autocite{architectural-pattern}. Dalam artian, \textit{Architectural Pattern} mirip dengan \textit{Design Pattern}, tetapi memiliki cakupan yang lebih luas\autocite{archi-pattern}.
\begin{itemize}
    \item \textit{Object-Oriented Architecture (OOA)}\\
          \textit{Object-Oriented Architecture} dapat digunakan jika ingin mengenkapsulasi logika dan data bersama-sama dalam komponen yang dapat digunakan kembali. \textit{Business Logic} yang kompleks yang membutuhkan abstraksi dan \textit{dynamic behaviour} dapat secara efektif menggunakan arsitektur jenis ini juga\autocite{architectural-pattern}.
    \item \textit{Layered/Tiered Architecture}\\
          \textit{Layered/Tiered Architecture} merupakan salah satu jenis \textit{architectural pattern} yang sangat sering digunakan\autocite{architectural-pattern}. Arsitektur jenis ini dapat digunakan untuk menyusun program yang dapat didekomposisikan menjadi beberapa kelompok \textit{subtasks}, yang masing-masing berada pada tingkat abstraksi tertentu. Setiap \textit{layer} menyediakan layanan \textit{(services)} ke \textit{layer} berikutnya yang lebih tinggi. Terdapat empat \textit{layer} utama didalam arsitektur jenis ini, yakni \textbf{\textit{Presentation Layer}}, \textbf{\textit{Application Layer}}, \textbf{\textit{Business Logic Layer}} dan \textbf{\textit{Data Access Layer}}. Arsitektur jenis ini biasanya digunakan pada aplikasi desktop dan \textit{web e-commerce} pada umumnya\autocite{archi-pattern}.
    \item \textit{Event-Driven Architecture (EDA)}\\
          \textit{Event-Driven Architecture (EDA)} biasanya didasarkan pada \textit{communication model} yang digerakkan oleh pesan asinkron untuk menyebarkan informasi ke seluruh organisasi/perusahaan. \textit{EDA} mendukung \textit{allignment} yang lebih alami dengan model operasional organisasi dengan menggambarkan \textit{business activity} nya sebagai suatu rangkaian \textit{event}/peristiwa\autocite{architectural-pattern}.
    \item \textit{Model-View-Controller Architecture (MVC)}\\
          \textit{Model-View-Controller Architecture} atau biasa juga disebut sebagai \textbf{MVC}, merupakan sebuah jenis \textit{architectural pattern} yang membagi sistem aplikasinya menjadi tiga bagian, yakni \textbf{\textit{Model}}, \textbf{\textit{View}} dan \textbf{\textit{Controller}}. \textbf{\textit{Model}} merupakan fungsi inti dari aplikasi dan menyimpan data, \textbf{\textit{View}} merupakan sebuah \textit{user interface (UI)} yang dapat dilihat atau berinteraksi dengan pengguna secara langsung dan \textbf{\textit{Controller}} sebagai penghubung antara \textit{model} dengan \textit{view} serta yang mengatur \textit{input} dari pengguna. Arsitektur jenis ini paling sering digunakan untuk membuat sebuah aplikasi berbasis \textit{web}\autocite{archi-pattern}.
\end{itemize}

\subsection*{2.4 \textit{Design Pattern}}
\addcontentsline{toc}{subsection}{\protect\numberline{}2.4 \textit{Design Pattern}}

\textit{Design Pattern} secara definisi adalah sesuatu cara komunikasi tiap obyek dan \textit{class} yang dikostumisasi untuk memecahkan masalah desain general didalam konteks tertentu, \textit{design pattern} membantu untuk membuat \textit{object oriented design} yang dapat digunakan secara berulang, dengan cara memberi nama, mengabstraksi, dna mengidentifikasi kunci aspek dari struktur desain tertentu\autocite{design-pattern-2588942}.
\begin{itemize}
    \item \textit{Facade}\\
          \textit{Facade} adalah sesuatu \textit{class} yang memberikan \textit{interface} yang simpel ke sub-sistem yang kompleks, \textit{design pattern} ini dapat digunakan ketika butuh mengintegrasikan sesuatu \textit{library} dengan fitur yang sangat banyak tapi hanya butuh menggunakan beberapa fiturnya saja\autocite{refactoring-guru}. \textit{Singleton} juga memberikan global \textit{access point} ke instansi \textit{class} tersebut, jadi semua \textit{class} dan \textit{object} dapat membaca \textit{value} yang sama dan merubah \textit{value} tersebut secara keseluruhan\autocite{refactoring-guru}.
    \item \textit{Singleton}\\
          \textit{Singleton} digunakan untuk menjamin bahwa suatu \textit{class} hanya memiliki satu \textit{instance}, \textit{design pattern} ini berguna saat ingin mengontrol sebuah \textit{shared resources} seperti \textit{database}\autocite{refactoring-guru}.
    \item \textit{Builder}\\
          \textit{Design pattern} ini dibuat untuk memecahkan masalah ketika ada sebuah \textit{class} yang saat dibuat menjadi sebuah \textit{object} tidak semua atributnya diisi, tanpa harus membuat \textit{contructor} atau \textit{child class} banyak. \textit{Builder} mengekstrasi \textit{building steps} saat membuat suatu \textit{object} dari sebuah \textit{class} dan memindahkannya ke \textit{class} terpisah\autocite{refactoring-guru}.
    \item \textit{Template Method}\\
          \textit{Template method} digunakan untuk mengurangi redudansi ketika ada beberapa fitur dengan langkah penyelesaian sama tapi dengan cara yang berbeda, \textit{template method} dibuat dengan cara memisahkan logika suatu fitur menjadi beberapa langkah dan saat ingin menambahkan fitur baru, bisa langsung membuat \textit{child class} dari \textit{template method} tersebut dan meng-\textit{override} step yang ingin diganti logikanya\autocite{refactoring-guru}.
\end{itemize}

\subsection*{2.5 Aplikasi Serupa}
\addcontentsline{toc}{subsection}{\protect\numberline{}2.5 Aplikasi Serupa}
Berikut adalah perbandingan antara beberapa aplikasi serupa dari hasil penelitian-penelitian terdahulu dan dua aplikasi pasaran yang sebelumnya sudah sedikit dibahas pada bagian latar belakang. Disini, aplikasi-aplikasi tersebut dapat dikategorikan menjadi dua jenis, yaitu \textit{marketplace} atau \textit{e-commerce} dan sebuah forum untuk berdiskusi secara general. Berikut adalah perbandingan kelebihan dan kekurangannya.
\textbf{Aplikasi-aplikasi berdasarkan penelitian terdahulu}:
\begin{longtable}{|m{3cm}|p{5cm}|p{5cm}|}
    \hline
    Aplikasi & Positif                                                                                                                 & Negatif \\
    \hline
    Application for Marketplace Agricultural Product
             & + Mempermudah para petani/pekebun untuk menjual belikan produk agrikulturnya \newline
    + User dapat me-\textit{request} produk yang belum ada agar bisa dijual di app tersebut \newline
             & - Hanya terfokus terhadap satu sektor saja yaitu agrikultur \newline
    - Tidak bisa berdiskusi langsung dengan penjual mengenai produk yg dijual \newline
    - Bukan \textit{marketplace} untuk hobi \newline
    - Hanya tersedia di satu kota saja \newline
    - Hanya bisa berjalan di Android saja, tidak bisa di IOS                                                                                     \\
    \hline
    e-Nelayan the Fishery Marketplace App
             & + Mempermudah para nelayan dan penjual ikan untuk menjual belikan ikan-ikannya \newline
    + User dapat me-\textit{request} ikan yang belum tersedia dengan membuat post agar bisa dilihat langsung oleh nelayan atau penjual ikan \newline
    + App ini dapat menghubungankan pihak user, penjual hingga nelayan melalui sistem comment pada postingan user \newline
             & - Hanya terfokus pada sektor perikanan saja \newline
    - Bukan \textit{marketplace} untuk hobi \newline
    - Hanya bisa berjalan di Android saja, tidak bisa di IOS                                                                                     \\
    \hline
    LIDI: A Web-Based Car Rent Marketplace Application
             & + Mempermudah user yang ingin menyewa mobil secara online \newline
    + Menampilkan informasi lengkap mengenai mobil yang disewakan \newline
             & - Hanya terfokus pada jasa sewa mobil \newline
    - Proses \textit{order} masih menggunakan antrian \textit{via} nomor telepon user \newline
    - User harus secara manual mengecek sendiri kondisi mobil yang ingin disewa \newline
    - Hanya tersedia di satu kota saja \newline
    - Bukan \textit{marketplace} khusus untuk hobi                                                                                               \\
    \hline
    e-Marketplace for Village-owned Small, Micro and Medium Enterprise
             & + Mempermudah user untuk mencari produk yang hanya dijual oleh bisnis-bisnis kecil \newline
    + Memepersatukan bisnis-bisnis kecil kedalam satu \textit{platform} online jual beli yang sama \newline
             & - \textit{User Interface} yang masih minim penjelasannya yang dapat membuat user kebingungan \newline
    - Hanya tersedia di satu lokasi saja \newline
    - Bukan \textit{marketplace} khusus untuk hobi                                                                                               \\
    \hline
    C2C marketplace model in fishery product trading application
             & + User dan nelayan dapat dengan mudah berinteraksi walaupun berada di pedesaan dengan infrastruktur yang jelek \newline
    + Sistem yang relatif simpel \newline
             & - Hanya terfokus pada sektor perikanan saja \newline
    - Hanya tersedia di satu lokasi saja \newline
    - Bukan \textit{marketplace} khusus untuk keperluan hobi                                                                                     \\
    \hline
\end{longtable}

\textbf{Aplikasi pasaran yang serupa dengan proyek ini}:
\begin{longtable}{|m{3cm}|p{5cm}|p{5cm}|}
    \hline
    Aplikasi & Positif                                                  & Negatif                \\
    \hline
    KyouId
             & + Bisa membeli barang dari negara lain (Jepang) \newline
    + Sudah integrasi langsung dengan banyak \textit{shipment service} yang ada di Indonesia \newline
             & - Hanya terfokus kesatu jenis hobi \newline
    - Tidak memiliki filter untuk \textit{interest level}                                        \\
    \hline
    Reddit
             & + \textit{community} yang sudah besar \newline
    + System handling \textit{post} dan komen sudah \textit{robust} \newline
    + Interaksi antara user sudah bagus
             & - Bukan marketplace \newline
    - Tidak cocok untuk memberi rating konkrit kepada barang \newline
    - Tidak ada gambar atau suatu panel untuk menampilkan informasi item tersebut secara singkat \\
    \hline
\end{longtable}

% \begin{longtable}{|m{2cm}|p{5cm}|p{5cm}|}
%     \hline
%     Aplikasi & Positif                                                                           & Negatif          \\
%     \hline
%     Reddit
%              & + \textit{community} yang sudah besar \newline
%     + System handling \textit{post} dan komen sudah \textit{robust} \newline
%     + Interaksi antara user sudah bagus
%              & - Bukan marketplace \newline
%     - Tidak cocok untuk memberi rating konkrit kepada barang \newline
%     - Tidak ada gambar atau suatu panel untuk menampilkan informasi item tersebut secara singkat                    \\
%     \hline
%     Tokopedia
%              & + Proses shipping dan payment sudah berbagai macam \newline
%     + Banyak promo yang bisa didapatkan user
%              & - Tidak bisa mengelompokkan item berdasarkan hobi \newline
%     - User akan sulit mencari item yang cocok bagi mereka karena ketidakadaannya interest level untuk item tersebut \newline
%     - Tidak ada forum dedikasi diskusi, hanya untuk diskusi barang                                                  \\
%     \hline
%     KyouId
%              & + Bisa membeli barang dari negara lain (jepang) \newline
%     + Sudah integrasi langsung dengan banyak \textit{shipment service} yang ada di Indonesia \newline
%              & - Hanya terfokus kesatu jenis hobi \newline
%     - Tidak memiliki filter untuk \textit{interest level}                                                           \\
%     \hline
%     Lazada
%              & + Sudah meng-\textit{support} berbagai macam cara pembarayan\newline
%     + Sudah integrasi langsung dengan banyak \textit{shipment service} yang ada di Indonesia \newline
%     + Banyak kategori barang yang sudah ada
%              & - Tidak bisa mengelompokkan item berdasarkan hobi \newline
%     - User akan sulit mencari item yang cocok bagi mereka karena ketidakadaannya interest level untuk item tersebut \\
%     \hline
%     Amazon
%              & + Memiliki \textit{internation shipping} hampir ke semua negara \newline
%     + Sudah memiliki support untuk berbagai macam mata uang \newline
%     + Miliki \textit{buyer protection} yang sangat bagus
%              & - Tidak memiliki filter yang bedasarkan hobi dan \textit{interest level} \newline
%     - Karena \textit{size} amazon yang sudah besar, banyak \textit{scammer} dan \textit{scalper} yang berkeliaran \newline
%     - Tempat yang disediakan untuk berdisuksi sangat terbatas                                                       \\
%     \hline
% \end{longtable}

% \subsection*{2.5 \textit{Website Mock}}
% \addcontentsline{toc}{subsection}{\protect\numberline{}2.5 \textit{Website Mock}}

% \subsubsection*{2.5.1 \textit{Buyer User Interface}}
% \addcontentsline{toc}{subsubsection}{\protect\numberline{}2.5.1 \textit{Buyer User Interface}}
% \begin{figure}[h]
%     \includegraphics[width=.30\textwidth]{ui/User View/Forum Discussion (All) - Post 1 [User].png}\hfill
%     \includegraphics[width=.30\textwidth]{ui/User View/Forum Discussions - All [User].png}\hfill
%     \includegraphics[width=.30\textwidth]{ui/User View/Forum Discussion - PS1 [User].png}\hfill
%     \\[\smallskipamount]
%     \includegraphics[width=.30\textwidth]{ui/User View/Item Page 1 [User].png}\hfill
%     \includegraphics[width=.30\textwidth]{ui/User View/Store Home Page [User].png}\hfill
%     \includegraphics[width=.30\textwidth]{ui/User View/Shopping Cart [User].png}\hfill
%     \caption{\textit{Mock-up user interface} untuk pembeli}
%     \centering
% \end{figure}

% \subsubsection*{2.5.2 \textit{Merchant User Interface}}
% \addcontentsline{toc}{subsubsection}{\protect\numberline{}2.5.2 \textit{Merchant User Interface}}
% \begin{figure}[h]
%     \includegraphics[width=.30\textwidth]{ui/Merchant View/All Chats [Merchant].png}\hfill
%     \includegraphics[width=.30\textwidth]{ui/Merchant View/Chat w User - Chat 1 [Merchant].png}\hfill
%     \includegraphics[width=.30\textwidth]{ui/Merchant View/Edit Profile [Merchant].png}\hfill
%     \\[\smallskipamount]
%     \includegraphics[width=.30\textwidth]{ui/Merchant View/Forum Discussions - All [Merchant].png}\hfill
%     \includegraphics[width=.30\textwidth]{ui/Merchant View/Product Management [Merchant].png}\hfill
%     \includegraphics[width=.30\textwidth]{ui/Merchant View/Store Home Page [Merchant].png}\hfill
%     \\[\smallskipamount]
%     \includegraphics[width=.30\textwidth]{ui/Merchant View/Update Product [Merchant].png}\hfill
%     \caption{\textit{Mock-up user interface} untuk penjual}
%     \centering
% \end{figure}

% \newpage
% \subsubsection*{2.5.3 \textit{Admin User Interface}}
% \addcontentsline{toc}{subsubsection}{\protect\numberline{}2.5.3 \textit{Admin User Interface}}
% \begin{figure}[h]
%     \includegraphics[width=.30\textwidth]{ui/Admin View/Admin Dashboard [Admin].png}\hfill
%     \includegraphics[width=.30\textwidth]{ui/Admin View/Approval Details [Admin].png}\hfill
%     \includegraphics[width=.30\textwidth]{ui/Admin View/Approval Page [Admin].png}\hfill
%     \\[\smallskipamount]
%     \includegraphics[width=.30\textwidth]{ui/Admin View/Reports Page - All [Admin].png}\hfill
%     \includegraphics[width=.30\textwidth]{ui/Admin View/Reports Page - Post [Admin].png}\hfill
%     \includegraphics[width=.30\textwidth]{ui/Admin View/Reports Page - Comment [Admin].png}\hfill
%     \\[\smallskipamount]
%     \includegraphics[width=.30\textwidth]{ui/Admin View/Store Home Page [Admin].png}\hfill
%     \caption{\textit{Mock-up user interface} untuk \textit{admin}}
%     \centering
% \end{figure}

\newpage
\section*{Bab 3. Metode Pelaksanaan}
\addcontentsline{toc}{section}{\protect\numberline{}Bab 3. Metode Pelaksanaan}
\subsection*{3.1 Metode Pengumpulan Data}
\addcontentsline{toc}{subsection}{\protect\numberline{}3.1 Metode Pengumpulan Data}


Pengumpulan data yang lakukan adalah dengan membuat form survei menggunakan \textit{Google Form} untuk mendapatkan pendapat atas pertanyaan-pertanyaan yang terkait dengan alasan dari pembuatan aplikasi pada proyek ini. Dari hal mengenai pembelian barang hobi hingga tingkat kesulitan dalam menemukan barang yang cocok. Dari semua jawaban survei yang dikumpulkan tersebutlah, dapat memperoleh gambaran dan konfirmasi atas fitur-fitur yang proyek ini butuhkan.


Pertanyaan “Apakah anda takut untuk mengambil hobi baru karena tidak tau mulai dari mana?”, diajukan kepada pengisi survei untuk mendapatkan data dan konfirmasi apakah aplikasi dalam proyek ini dibutuhkan, dan dengan hasil jawaban mayoritas diatas 60\% menjawab iya, dari data tersebut bisa disimpulkan adanya kesulitan memulai hobi karena keraguan titik mulai hobi tersebut.


Selanjutnya dengan “Apakah anda sering menggunakan \textit{e-commerce} untuk browsing dan mencari barang berhubungan dengan hobi anda?” untuk validasi apakah menggunakkan fasilitas online untuk membeli barang hobi itu merupakan metode yang populer. Dengan masukan yang didapatkan dari survei, 70\% menjawab iya maka dapat disimpulkan bahwa fasilitas jual/beli barang hobi di internet itu layak dilakukan.


Pertanyaan “Dalam mencari dan menemukan barang apakah anda terkadang merasa ragu dan harus mencari tahu terlebih dahulu atas barang yang anda lihat apakah untuk pemula/menengah/ahli?” ini adalah untuk mengumpulkan data atas fitur utama yang dirancang yaitu fitur filtrasi level peminatan(\textit{interest level}) dari suatu hobi Dengan respons yang berjawab iya diatas 70\% maka fitur utama dari proyek ini layak untuk diimplementasikan.

% \subsection*{3.2 Analisis}
% \addcontentsline{toc}{subsection}{\protect\numberline{}3.2 Analisis}

% \subsubsection*{3.2.1 Analisis perbandingan aplikasi sejenis}
% \addcontentsline{toc}{subsubsection}{\protect\numberline{}3.2.1 Analisis perbandingan aplikasi sejenis}
% Terdapat dua aplikasi serupa dengan aplikasi didalam proposal proyek ini yang kami jadi obyek oberservasi, yaitu \textit{Reddit} dan \textit{Tokopedia}.

% \newpage
% \begin{figure}[h]
%     \begin{longtable}{|m{2cm}|p{5cm}|p{5cm}|}
%         \hline
%         Aplikasi & Positif                                                     & Negatif             \\
%         \hline
%         Reddit
%                  & + \textit{community} yang sudah besar \newline
%         + System handling \textit{post} dan komen sudah \textit{robust} \newline
%         + Interaksi antara user sudah bagus
%                  & - Bukan marketplace \newline
%         - Tidak cocok untuk memberi rating konkrit kepada barang \newline
%         - Tidak ada gambar atau suatu panel untuk menampilkan informasi item tersebut secara singkat \\
%         \hline
%         Tokopedia
%                  & + Proses shipping dan payment sudah berbagai macam \newline
%         + Banyak promo yang bisa didapatkan user
%                  & - Tidak bisa mengelompokkan item berdasarkan hobi \newline
%         - User akan sulit mencari item yang cocok bagi mereka karena ketidakadaannya interest level untuk item tersebut \newline
%         - Tidak ada forum dedikasi diskusi, hanya untuk diskusi barang                               \\
%         \hline
%     \end{longtable}
%     \caption{Tabel perbandingan aplikasi serupa}
% \end{figure}

% \subsubsection*{3.2.2 Analisis permasalah}
% \addcontentsline{toc}{subsubsection}{\protect\numberline{}3.2.2 Analisis permasalah}
% Salah satu pertanyaan adalah "Apakah anda takut untuk mengambil hobi baru karena tidak tau mulai dari mana?" dan dengan responden menjawab 69.2\% iya mereka takut untuk mengambil hobi baru karena alasan yang disebut, beberapa alasan kenapa banyak orang takut untuk mengambil hobi baru adalah,
% \begin{itemize}
%     \item mereka tidak tahu apa yang harus dibeli sebagai \textit{beginner};
%     \item mereka tidak tahu apa yang mereka beli cocok untuk \textit{interest level} mereka atau tidak;
%     \item mereka tidak tahu hal apa saya yang perlu diketahui untuk memasuki hobi tersebut.
% \end{itemize}
% Dengan 61.5\% menjawab mereka tidak punya waktu untuk mencari tahu informasi tentang hobi tersebut, satu masalah muncul yaitu "bagaimana cara mempercepat proses pemilihan barang untuk pemula hobi".

% \subsubsection*{3.2.3 Usulan pemecahan masalah}
% \addcontentsline{toc}{subsubsection}{\protect\numberline{}3.2.3 Usulan pemecahan masalah}
% Untuk orang yang ingin memulai hobi dan mempunyai banyak waktu untuk mencari informasi tentang hobi tersebut, solusi kami adalah membuat sebuah \textit{marketplace} dengan sebuah \textit{page} yang khusus dibuat untuk para pengguna berdiskusi, berdiskusi tentang sebuah \textit{product} dan ke siapa saja \textit{product} ditujukan.


% Untuk orang yang tidak punya waktu untuk melihat semua itu, \textit{project} ini akan menerapkan dua sistem \textit{filter}, \textit{rating} sebuah \textit{product} yang menandakan seberapa berkualitas \textit{product} tersebut dan \textit{rating interest level} dari \textit{product} yang menandakan \textit{product} tersebut itu diarahkan ke siapa. Sistem \textit{interest level rating} ini akan memiliki dua value, \textit{value} pertama adalah \textit{value} yang ditetapkan oleh penjual dan \textit{value} kedua itu yang ditetapkan oleh komunitas, jadi secara tidak langsung jika ada barang yang dengan kedua \textit{rating}nya tidak sama, sipembeli bisa menyimpulkan bahwa penjual ini tidak bisa dipercaya.


\subsection*{3.2 Metode Pengembangan Sistem}
\addcontentsline{toc}{subsection}{\protect\numberline{}3.2 Metode Pengembangan Sistem}

\subsubsection*{3.2.1 \textit{Software Design Document}}
\addcontentsline{toc}{subsubsection}{\protect\numberline{}3.2.1 \textit{Software Design Document}}
\begin{enumerate}[label=\alph*. ]
    \item Deskripsi \textit{software}

          \textit{Shumishumi} adalah sebuah marketplace dimana user bisa membeli suatu barang sesuai dengan hobi dan setinggi apa \textit{interest level} mereka. Tujuan dari aplikasi ini adalah mempermudah pengguna pembeli untuk mencari barang hobi yang cocok, mempermudah penjual dalam mempromosikan barang-nya kepada pelanggan dengan fitur filter \textit{Interest Level}.


          \textit{Shumishumi} juga bisa menjadi sarana untuk berdiskusi mengenai topik dan hobi yang diminati, dengan fitur ini \textit{user} diharapkan dapat mencari hobi baru dengan mudah.

    \item Fungsi-fungsi \textit{software}


          Fungsi dari proyek ini dibagi menjadi dua perspektif, perspektif dari pembeli dan perspektif dari penjual.
          \begin{itemize}
              \item Perspektif dari pembeli

                    Pembeli bisa mencari sebuah barang secara spesifik menggunakan nama atau siapa yang menjual barang tersebut, tapi bagi penjual yang baru ingin memasuki suatu hobi, mereka bisa mengutilisasikan fitur filter bedasarkan hobi apa yang ingin mereka masuki dan \textit{interest level} mereka. Selain itu juga pembeli juga bisa membuat review tentang barang yang mereka jual, bagaimana kualitas dari barang tersebut dan apakah barang tersebut memenuhi kriteria \textit{interest level} yang ditujukan.

                    Selain itu juga pembeli bisa membuat suatu \textit{post} atau \textit{comment} tentang hobi mereka, atau membaca \textit{post} yang sudah dibuat oleh pembeli lain untuk membantu mereka mendalami hobi yang mereka punya.

              \item Perspektif dari penjual

                    Penjual bisa menjual barang, saat memasukan barang yang ingin dijual penjual harus memasukan barang itu terkait dengan hobi apa dan ditujukkan ke \textit{interest level} mana, selain itu penjual juga bisa melihat sebuah \textit{dashboard} yang berisikan \textit{review} dari pembeli dan berapa banyak barang yang sudah mereka jual.
          \end{itemize}

          % sequence diagram

    \item Kebutuhan teknologi

          \begin{enumerate}
              \item \textit{Software Development Life-Cycle - Agile}\\
                    \textbf{SDLC} yang dipakai dalam proyek ini adalah \textit{Agile}, dikarenakan \textit{agile} dapat membantu proyek ini dikerjakan dalam \textit{size} yang lebih kecil dan bisa diwujudkan dengan \textit{timeframe} yang masuk akal dan bersifat \textit{continuous} jadi lebih fleksible saat menghadapi perubahan \textit{requirement}\autocite{atlassian-agile}.

              \item \textit{Back end}

                    \textit{Back end} atau bisa disebut juga \textit{developer's end} adalah sebuah layer yang memproses semuanya dibelakang layar dan tempat terjadinya sesuatu yang tidak bisa dilihat oleh \textit{user}\autocite{letsgodojo-frontend-backend}. Analogy yang bisa digunakan adalah saat disebuah restoran ada pelanggan yang memesan sesuatu yang spesifik ke pelayan, ini adalah bagian \textit{front end} dari restoran tersebut, setelah itu pelayan memberikan pesanan tersebut ke koki, yang akan mengambil bahan masak, momotong semua bahan - bahan tersebut, dan memasaknya sesuai resep yang sudah dibuat, koki ini yang merepresentasikan bagian \textit{backend} dari restoran tersebut\autocite{codecademy-backend}.

                    \textit{Back end} terdiri dari dua hal, \textit{servers} yang akan meproses data dan \textit{request} dari sebuah user dan \textit{database} yang akan menyimpan data yang sudah atau yang akan mau diproses\autocite{codecademy-backend}. Proyek ini akan menggunakan \textit{springboot} sebagai \textit{framework} yang digunakan untuk membuat \textit{software back end}-nya dan \textit{\textbf{MySQL}} sebagai database yang digunakan.

                    \textit{Springboot} adalah \textit{java back end framework} yang paling populer didunia, \textit{springboot} membuat menulis \textit{software back end} menjadi lebih mudah dan lebih cepat\autocite{spring-framework}, alasan kenapa proyek ini menggunakan \textit{springboot} adalah sebagai berikut:
                    \begin{itemize}
                        \item \textit{Springboot} mempunyai banyak \textit{plugin} yang dapat di\textit{install};
                        \item komunitas \textit{springboot} itu sangat besar;
                        \item dan yang terakhir \textit{springboot} memiliki fitur \textit{Inversion of Control} dan \textit{Dependency Injection}.
                    \end{itemize}

              \item \textit{Front end}

                    \textit{Front End} adalah sebuah bagian dari halaman web yang dimana merupakan tempat interaksi pengguna, \textit{Front End} sebagai tempat interaksi pengguna maka merupakan bagian dari sebuah halaman yang sangat sering dilihat oleh pengguna kebalikan dari \textit{Back End}\autocite{codecademy-frontend}.

                    \textit{Javascript} adalah bahasa pemrograman terkompilasi yang memiliki karakteristik ringan untuk dijalankan, \textit{interpreted} dan juga dengan fitur \textit{first-class functions} dimana \textit{function} dianggap seperti variabel biasa lainnya seperti \textit{string, number,} dan lain-lain. Yang dimana penggunaan \textit{Javascript} paling terkenal dan banyak digunakkan adalah dalam bahasa untuk \textit{Web} atau situs internet\autocite{javascript-mdn}.

                    \textit{Typescript} adalah bahasa pemrograman yang dibuat untuk memecahkan beberapa permasalahan yang ada di dalam bahasa \textit{Javascript}, salah satu contohnya adalah dengan \textit{Typescript} penggunaan \textit{Type-annotation} dapat digunakkan untuk memberi tipe dari variabel sehingga permasalahan yang mungkin dapat muncul karena karakteristik \textit{Javascript} yang \textit{loosely typed} dapat diatasi\autocite{fenton2014pro}.

                    \textit{React} adalah sebuah Library untuk bahasa pemrograman \textit{Javascript} (\textit{Typescript} juga bisa) untuk membuat \textit{UI}. Fitur dari \textit{React} adalah \textit{React} menganut \textit{Declarative Programming} dan juga \textit{Functional Programming},\textit{Component Based}, \textit{Hooks} dan \textit{Lifecycle Hooks}\autocite{react-general}. \textit{Hooks} dan \textit{Lifecycle Hooks} adalah sebuah fitur yang diimplementasikan pada \textit{React v16.8} yang memberi alternatif dari menggunakkan \textit{class} untuk komponen\autocite{react-hooks, react-hooks-lifecycle}.

          \end{enumerate}

\end{enumerate}


\newpage
\subsubsection*{3.2.2 \textit{Timeline}}
\addcontentsline{toc}{subsubsection}{\protect\numberline{}3.2.2 \textit{Timeline}}

Karena proyek ini menggunakan \textit{agile} sebagai \textbf{SDLC} yang digunakan, maka \textit{timeline} ini dipisah menjadi satu bulan pertama untuk \textit{planning} dan lima bulan yangd dipisah menjadi lima \textit{sprint} yang berbeda. Dalam satu \textit{sprint} akan dipisah lagi menjadi satu minggu pertama untuk melakukan \textit{spring planning}, dua minggu untuk masa \textit{development frontend} dan \textit{backend} secara bersamaan dan satu minggu akan dipakai untuk masa \textit{testing} untuk memastikan fitur yang di\textit{develop} tidak menggangu fitur yang sudah ada dan bisa digunakan.

\begin{figure}[h]
    \includegraphics[width=13cm]{sprint timeline.png}
    \centering
    \caption{\textit{Timeline} untuk bulan maret sampai mei.}
\end{figure}

\begin{figure}[h]
    \includegraphics[width=13cm]{sprint timeline 2.png}
    \centering
    \caption{\textit{Timeline} untuk bulan juni sampai agustus.}
\end{figure}

\newpage


% \subsubsection*{3.2.2 Perancangan Sistem}
% \addcontentsline{toc}{subsubsection}{\protect\numberline{}3.2.2 Perancangan Sistem}
% \begin{enumerate}
%     \item Desain \textit{database}\\
%     \begin{figure}[h]
%         \includegraphics[height=12cm,keepaspectratio]{erd diagram.png}
%         \centering
%         \caption{\textit{Entity Relation Diagram}}
%     \end{figure}
%     \newpage
%     \item \textit{Sequence Diagram}\\
%     \begin{itemize}
%         \item \textit{Buyer flow sequence diagram}\\
%             \begin{figure}[h]
%                 \includegraphics[height=12cm,keepaspectratio]{sequence diagram user perspective.png}
%                 \centering
%                 \caption{\textit{Sequence diagram} dari perspektif pembeli}
%             \end{figure}
%         \newpage
%         \item \textit{Merchant flow sequence diagram}
%             \begin{figure}[h]
%                 \includegraphics[height=12cm,keepaspectratio]{sequence diagram merchant perspective.png}
%                 \centering
%                 \caption{\textit{Sequence diagram} dari perspektif penjual}
%             \end{figure}
%         \newpage
%         \item \textit{Admin flow sequence diagram}
%             \begin{figure}[h]
%                 \includegraphics[height=12cm,keepaspectratio]{sequence diagram admin perspective.png}
%                 \centering
%                 \caption{\textit{Sequence diagram} dari perspektif \textit{admin}}
%             \end{figure}
%     \end{itemize}
% \end{enumerate}

\newpage
\addcontentsline{toc}{section}{\protect\numberline{}Referensi}
\printbibliography[title=Referensi]

\end{document}
